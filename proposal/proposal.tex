\documentclass{article}
\usepackage{enumerate}
\usepackage{calc}

\usepackage{hyperref}
\usepackage[T1]{fontenc}
\usepackage[top=1.0in, bottom=1.0in, left=1.0in, right=1.0in]{geometry}
\pagenumbering{gobble}
\usepackage{xcolor}
\usepackage{titlesec}
\titleformat{\section}
  {\normalfont\normalsize\bfseries}
  {\thesection}{1em}{}
\titlespacing*{\section}{0pt}{0.4\baselineskip}{0.3\baselineskip}

\begin{document}

\begin{center}
{\Large\bf 6.883 Final Project Proposal:} \\[1ex]
{\Large\bf Mind-Reading Machine++} \\[1ex]
{\large John Holliman (holliman@mit.edu)}
\end{center}

\vspace{-12pt}
\noindent\makebox[\linewidth]{\rule{\textwidth}{1pt}}
\vspace{3pt}

\section*{Overview}

In the 1950's, as described in lecture 1 of class, David Hagelbarger and Claude
Shannon built the first "mind reading" machines to play the game of matching
pennies \cite{Hag56, Sha53}.  Rumor has it that Shannon's bot beat
Hagelbarger's 55 to 45. A modern version of a "mind reading" machine
(\url{http://www.mindreaderpro.appspot.com}) was produced by Professor Yoav
Freund and his colleagues at UC Berkeley. Professor Freund's machine uses
hedging algorithms and context-weighted trees.

In this project, I will produce my own "mind reading" machine that explores
using expert's advice, as described on
\url{http://www.mindreaderpro.appspot.com/about} as future work.

In short, I would like to:
\begin{enumerate}
	\item Familiarize myself with the relevant past and current work \cite{Hag56,
		Sha53, Willems95, CB97} on predicting sequences of bits
	\item Implement various `experts' and combine them into a single predictor
	\item Compare my implementation against those of Hagelbarger, Shannon, and Freund.
\end{enumerate}

\medskip

% references section
\bibliographystyle{unsrt}
\bibliography{sources}

\end{document}
